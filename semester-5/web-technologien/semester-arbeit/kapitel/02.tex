\newpage
\section{Konzepte in JavaScript Frameworks} \label{konzepte}

Bevor in dieser Arbeit konkrete Frameworks vorgestellt werden, müssen einige
Begriffe und Konzepte erklärt werden. Es wird uns später helfen, eine
Klassifizierung vorzunehmen.

\subsection{\acl{SPA}}

Eine \acr{SPA} ist eine Web Applikation oder Website, die dynamisch die Inhalte
ändert, ohne auf eine neue Website umzuleiten. Ein Nutzer interagiert mit der
Seite, woraufhin client-seitiges JavaScript die Website verändert (durch
Hinzufügen, Verändern oder Entfernen von \acs{DOM}-Elementen).

Eine typische Interaktion ist in Abbildung \ref{fig:spa-interactions} illustriert.
Beim ersten Aufruf der Website, werden HTML, CSS und JavaScript Dateien an den
Client gesendet. Der Client rendert dann die Website für den Benutzer.
Wenn der User mit der Website interagiert kann es sein, dass neue Daten vom Server
abgefragt werden müssen. Anstatt auf eine neue Website zu navigieren, sendet der
Client ein Fetch - die neue Version der \ac{XHR} \ac{API} - an den Server (oder eine andere
Backend API), und bekommt eine Antwort, die der client-seitige JavaScript Code
interpretiert. Danach wird das DOM entsprechend angepasst.

\begin{figure}[H]
  \centering
  \caption[]{Interaktionen zwischen Client und Server bei SPAs}
  \label{fig:spa-interactions}
  \begin{sequencediagram}
    \newthread{A}{Client}{}
    \newinst[3]{B}{Website}{}

    \begin{call}{A}{Initialer Aufruf}{B}{HTML und JavaScript}
    \end{call}

    \begin{call}{A}{Fetch API/XHR}{B}{JSON/plain text/HTML}
    \end{call}
  \end{sequencediagram}
\end{figure}

Ein Vorteil von dieser Vorgehensweise ist, dass, obwohl der Initiale Aufruf
recht lange dauert, alle weiteren Änderungen recht schnell möglich sind.

- Konzepte
  - SPA
  - PolyFill und babel/webpack
  - Virtual DOM
  - Opinionated
  - Two way data binding
  - Custom Components
  - SSR
  - MVC
  - TypeScript / JavaScript / JSX
