\newpage
\section{Einleitung} \label{Einleitung}

JavaScript ist eine Sprache, die häufig im Web genutzt wird, um dynamische
Websiten zu erstellen. Wenn JavaScript Dateien in eine Website eingebunden
werden, werden diese Clientseitig ausgeführt. JavaScript bietet unter Anderem
Anweisungen zur Manipulation des \acr{DOM}, Bearbeitung des lokalen Speichers und dem
Aufrufen von anderen Websiten oder \acr{API}.

Wenn eine dynamische Website eine gewisse Größe (und somit Komplexität)
erreicht, wird es zunehmend aufwendiger den Code zu pflegen. Eine große Hürde
ist die Vermischung von verschiedenen Abstraktionsebenen: Auf der einen Hand
müssen low-level Operationen wie DOM Manipulation programmiert werden, auf der
anderen Hand werden high-level Benutzer Interaktionen designt und implementiert.

Aus diesen Problemen sind JavaScript Front-End Frameworks entstanden, die die
Erstellung von dynamischen Websiten vereinfachen sollen. In dieser Arbeit wird
aufgezeigt welche Frameworks verbreitet sind, wie sie kategorisiert werden
können und nach welchen Paradigmen sie funktionieren.

\subsection{Forschungsfrage}

Welche JavaScript Frameworks existieren, welche Funktionen bieten sie und wie
sieht eine Beispiel-Applikation aus?

\subsection{Aufbau der Arbeit}

- Konzepte
  - SPA
  - PolyFill und babel/webpack
  - DOM und Virtual DOM
  - Opinionated
  - Two way data binding
  - Custom Components
  - SSR
  - MVC
  - TypeScript / JavaScript / JSX
- Sprachen und Codebeispiele
  - Siehe 03.tex
  - Codebeispiel:
    - Excel-Clone
    - TODO Liste
