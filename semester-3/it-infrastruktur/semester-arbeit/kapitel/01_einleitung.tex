\newpage
\section{Einleitung} \label{Einleitung}
Ein wichtiger Erfolgfaktor für Unternehmen ist es aktuelle Trends zu erkennen
und ihre Strategie entsprechend anzupassen. Damit nicht jedes Unternehmen selbts
die Marktforschung vornehmen muss, haben sich einige Unternehmen als Ziel
genommen, die Analyse vorzunehmen und bereitszustellen. 


\subsection{Forschungsfrage}
Wie verlässlich sind die Analysen von kommenden Trends?

\subsection{Aufbau der Arbeit}

Im Kapitel \ref{analysten} werden einige Unternehmen vorgestellt, die IT-Trends
analysieren und bewerten.

Im Kapitel \ref{infrastruktur} wird die Entwicklung von Hybrid Cloud und die
aktuelle Unterstützung von den großen Cloud-Anbietern (\ac{GCP}, Microsoft Azure und
\ac{AWS}) aufgezeigt.

Im Kapitel \ref{verlaesslichkeit} werden die Unternehmen auf Grundlage ihrer
Einschätzung zu Hybrid Cloud bewertet. Ebenfals wird in diesem Kapitel gezeigt,
wie man einen solchen Trend frühzeitig erkennen kann.

Zuletzt werden die Ergebnisse dieser Semesterarbeit im Kapitel \ref{fazit}
zusammengefasst.

\subsection{Abgrenzung der Arbeit}
Die 

\subsection{Unterschied zwischen Trend und Hype}

In dieser Arbeit wird zwischen Trend und Hype unterschieden. Ein Trend
bezeichnet eine neue Technologie oder Arbeitsweise, die vielen Unternehmen einen
Wettberwerbsvorteil verschaffen - oder notwendig sind, um zumindest keinen
Wettbewerbsnachteil zu haben.

Ein Hype hingegen ist eine neue Technologie oder Arbeitsweise, die eine kurze
Zeit sehr viel Aufmerksamkeit durch Unternehmen und Gesellschaft bekommt, aber
wenig oder keinen nachhaltigen Mehrwert für Unternehmen bietet. Der Hype stribt
nach kurzer Zeit aus (siehe auch Abbildung \ref{fig:trend_vs_hype}).

\begin{figure}[H]
\caption{Trend vs. Hype}\label{fig:trend_vs_hype}
\begin{tikzpicture}[xscale=10,yscale=5]
% Achsen
\draw[<->] 
     (1,0)
  -- (0.5, 0) node[below]{Zeit}
  -- (0,0) 
  -- (0,0.5) node[left]{Erwarteter Erfolg}
  -- (0,1);

%hype
\draw[domain=0:0.3778,samples=100] plot (\x,{0.95*2^(-(11.9*(\x-0.3))^2)+0.04});
\draw[domain=0.3778:1,samples=100,orange] plot (\x,{0.95*2^(-(11.9*(\x-0.3))^2)+0.04});
\draw (0.6,0.05) node[above]{Hype};
%trend
\draw[blue]
  (0.3778,0.564166) to [out=-82.88,in=-180] (0.58,0.37) to [out=0,in=-150] (1,0.7);
\draw (0.6,0.5) node[below]{Trend};

\end{tikzpicture}
\\
Quelle: Eigene Darstellung basierend auf Gartner Hype-Cycle
\end{figure}
