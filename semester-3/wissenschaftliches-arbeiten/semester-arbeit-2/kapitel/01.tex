\newpage
\section{Einleitung} \label{Einleitung}

Ein Trend in der Softwarearchitektur ist aktuell die
\ac{MSA}\footnote{https://www.oreilly.com/radar/microservices-adoption-in-2020/}.
Dabei werden verschiedene Komponenten, Services und Subsysteme einer Applikation
identifiziert und seperat entwickelt. Es gibt eine Vielzahl an
wissenschaftlichen Arbeiten aus der Industrie, die sich damit befassen, wie eine
Migration von Monolithen zu einer MSA durchgeführt werden kann, doch gibt es
wenig zu finden, um die Frage zu beantworten, wann und zu welchen Umständen eine
solche Migration Sinn ergibt. Diese Arbeit fasst den aktuellen Forschungsstand
und Erfahrungsberichte aus der Industrie zusammen, um eine Antwort auf diese
Frage zu finden.

\subsection{Forschungsfrage}
Unter welchen Umständen ergibt eine Migration von Monolithen zu Microservices Sinn?

\subsection{Aufbau der Arbeit}
Im Kapitel \ref{msa} wird zunächst die MSA erklärt und in die allgemeinen
Software Architektur eingeordnet. Auch wird in diesem Kapitel die Monolithische
Architektur vorgestellt und mit der MSA kontrastiert.

Kapitel \ref{wann-msa} befasst sich mit der Definition von MSA und wie
sie die Probleme aus dem vorherigen Kapitel lösen. Ebenfalls wird kurz darauf
eingegangen, warum die MSA in der heutigen Zeit ein erhöhte Adoption findet.

Im Kapitel \ref{wann-nicht-msa} werden die Grenzen der MSA aufgezeigt und wann
es keinen wirtschaftlichen Sinn ergibt, eine Applikation in einer MSA zu entwickeln.

Zuletzt werden die Ergebnisse dieser Semesterarbeit im Kapitel \ref{fazit}
zusammengefasst.
