\newpage
\section{Vorraussetzungen für die Nutzung von Microservice Software Architekturen} \label{wann-msa}

Viele der in diesem Kapitel genannten Vorraussetzungen sind keine harten
Vorraussetzungen (dh. sie müssen erfüllt sein), sondern eher best-practices aus
Management und Softwareentwicklung, die eine Adoption einer MSA deutlich
vereinfacht und erfolgreich werden lässt.

\subsection{Anforderungen an die Organisation}

Im Jahre 1967 hat der Informatiker Melvin E. Conway das Paper "How Do Commitees
Invent?" geschrieben, das postuliert, dass Organisationen Systeme entwickeln,
die eine Kopie der Kommunikationsstruktur der Organisaiton ist. Diese These
wurde durch Fred Brooks' Buch "The Mythical Man-Month" als Conway's Gesetz
bekannt und wurde in 2007 durch eine quantitative Studie der Harvard Business
School belegt.\footcite{MACCORMACK20121309} Die Studie fand heraus, dass
Organisationen, die lose miteinander gekoppelt sind, modularere Design
entwickeln als es Organisationen mit einer engen Kopplung machen.

Es ist folglich nicht möglich über die Architektur eines Sytems zu reden, ohe
auch die organisatorischen Anforderungen an diese Architektur zu erwähnen. Um
MSA (dh. ein modulares System) zu entwickeln, muss die Organisation lose
gekoppelt sein. Wie soetwas aussehen kann und wie man MSA effektiv nutzt, zeigt
dieses Unterkapitel.

\subsubsection{Die Größe von Teams}

Die kleinste Eintelung innerhalb einer Organisation ist das Team. Dies ist
belegt durch Studien wie \cite{doi:10.1177/001872089203400303}. Eine zentrale
Frage ist nun, wie groß ein Team sein sollte, damit es effektiv arbeiten kann.

Ein Begriff, der häufig im Anwendungsentwicklungskontext auftaucht, ist das Two
Pizza Team. Er wurde von Amazon eingeführt und bedeutet, dass Teams so groß sein
sollten, dass sie von zwei Pizzen satt werden
können\footnote{https://docs.aws.amazon.com/whitepapers/latest/introduction-devops-aws/introduction-devops-aws.pdf#two-pizza-teams}.
Es ist bereit in der Definition bemerkbar, dass es mehr eine Richtlinie als
eine Regel ist.

Eine andere Heransgehensweise mit dem selben Ergebnis ist die kognitive Last auf
Teammitglieder genauer zu betrachten. Im Buch "Team
Topologies"\footnote{reference} wird Dunbar's
number\footnote{https://doi.org/10.1016/0047-2484(92)90081-J} genutzt, um die
effektivste Teamgröße zu bestimmen. Zwischen ? und ?? Menschen pro Team mit
Größen der nächst höheren Gruppierungen mit ??, ?? und ??? Menschen.

\subsubsection{Arten von Teams}

Ebenfalls im Buch "Team Topologies" werden Teams in 4 Grundsätzliche Arten
unterteilt.

\begin{enumerate}
  \item Stream-aligned Teams\label{team-art-stream}
  \item Complicated Subsystem Teams\label{team-art-complicated}
  \item Enabling Teams\label{team-art-enabling}
  \item Platform Teams\label{team-art-platform}
\end{enumerate}

Stream-aligned Teams sind Teams, die an die Business Funktion ausgerichtet sind.
In diesen kollaborieren die Mitglieder eng mit Experten aus dem Business Kontext
zusammen, um eine möglichst passende Lösung für das Unternehmen herzustellen.
Alle anderen Team-Arten existieren, um diesen Teams zu helfen.

Complicated subsystem Teams kümmern sich um Untersysteme, die schwierig für Outsider
zu verstehen sind. DIese Untersysteme benötigen spezielles Wissen und Expertise.
DIe Stream-aligned Teams in dieses System einzuarbeiten würde zu hohen
kognitiven Overhead verursachen, um sich auch noch den alltäglichen Aufgaben zu
widmen. In der Praxis könnte sich ein Complicates Subsystem Team um Sachen wie
Monolithen aus früheren Zeiten, oder business-critical Applikationen kümmern.

Enabling Teams helfen anderen Teams, neue Technologien zu verstehen und
einzusetzen. Der FOkus in diesen Team ist Kollaboration mit anderen Teams,
Innovation und das vorleben von Continuous Learning.

Zuletzt sind die Platform Teams dafür zuständig eine Plattform für anderen Teams
bereitzustellen. Es setzt Standards und entwickelt Frameworks. Es sind quasi
Service Provider für die anderen Teams.

Es ist bemerktbar, dass hier keine Teamaufteilung nach Technologien erfolgt
(bsp. Linux-Team, Datenbank-Team, Entwicklungs-Team, etc.). Die Autoren von
"Team Topologies" halten es für wichtig, dass alle Kompetenzen für die
Erbringung eines Services innerhalb eines Teams ist. Das überscheidet
sich mit der Microservice-Definition nach Fowler und Lewis - vor allem mit dem Punkt
"Produkte, nicht Projekte". Dadurch, dass es keine Unterteilung zwischen
Entwicklung und Betrieb einer Applikation gibt, soll es einfacher werden für ein
Team schneller zu agieren. Ebenso ist es einfacher zu skalieren, weil neue
Funktionalitäten durch ein neues Team gebaut und betrieben werden können,
anstatt existierenden Teams noch mehr Arbeit zu geben.

\subsection{Die Bedeutung von Domain Driven Design}

Mit dem Wissen über Team-größe und Arten ist nun wichtig, an welchen Kriterien
die Teams aufgeteilt werden. Im vorherigen Unterkapitel wurden Stream-aligned
Teams als Teams beschrieben, die an der Unternehmensfunktion ausgerichtet sind.
In diesem Unterkapitel wird gezeigt, wie eine solche Einteilung erfolgen kann,
und warum es so wichtig ist, wenn eine MSA eingeführt werden soll.

Domain-Driven Design ist eine Möglichkeit Software zu analysieren und zu planen.
Domain-driven Design wurde im Buch "Domain-Driven Design: Tackling Complexity in
the Heat of Software" von Eric Evans im Jar 2004 beschrieben.

Evans nennt drei Prinzipien, auf denen Domain-driven Design basiert:

\begin{enumerate}
  \item Fokus auf der Domäne und der Domänenlogik
  \item Basiere komplexe Designs auf Modellen der Domäne
  \item Kollaboriere mit Domänen Experten
\end{enumerate}

Eine Domäne in diesem Fall ist ein abgeschlossener Bereich innerhalb der
Unternehmenslogik. Der Vorgang das Domain-driven Designs ist es diese Bereiche
einzugrenzen und innerhalb der Bereiche eine gemeinsame Sprache zwischen
Domänenexperten und Anwendungsentwicklern zu entwerfen. Wenn die Domänen
analysiert und abgegrenzt wurden, können Stream-aligned Teams zu diesen gemappt
werden.

Ein weiterer Bestandteil des Domain-driven Designs ist die Entwicklung einer
Sprache zwischen verschiedenen Domänen. Diese wird genutzt, wenn Services Daten
untereinander austauschen müssen. Dies kann über Modelle wie eine Context Map
passieren. In solch einer Modellierung wird deutlich, welche Sprache zwischen
den Domänen genutzt wird; Was ist ein Kunde, wie sehen Bestellungen aus, ...

Evans argumentiert, dass Domain-Driven Design und Microservices gut zueinander
passen, weil Domain-Driven Design eine klare Abtrennung zwischen Domänen
benötigt, und dieser durch einer MSA gegeben werden kann.


% https://www.geeksforgeeks.org/domain-driven-design-ddd/
% https://github.com/ddd-by-examples/library
% https://www.codeproject.com/Articles/1094774/Domain-Driven-Design-A-hands-on-Example-Part-of-3

\subsection{Technologische Anforderungen}

Es gibt einen Grund, warum MSA eine erhöhte Popularität genießen. Es gibt
technologische Fortschritte und best-practices, durch die ein hoher
Automatisierungsgrad erreicht werden kann.

In dieser Arbeit werden zwei dieser Aspekte hervorgehoben: Cloud und
Containerisierung.

\subsubsection{Automatisierung durch Cloud}

Das NIST definiert Cloud Computing nach einigen essentiellen
Kriterien.\footcite{nistcloudcomputing} Das erste und für diese Arbeit einzig
wichtige Kriterium lautet on-demand self-service. Es beschreibt, dass ein Nutzer
der Cloud Resourcen wie Rechenleistung, Netzwerke und persistenten Speicher
selbst erstellen und verwalten kann. Diese Möglichkeit steht dem traditionellen
Datenzentrum entgegen, in dem durch interne Prozesse die Provisionierung von
diesen Resourcen Wochen bis Monate dauern kann. Diese Flexibilität ermöglicht
es, dass Teams Services relativ schnell erstellen und ändern können.

Ebenfalls bieten viele Cloud Computing Anbieter eine API an, die genutzt werden
kann, um Cloud Resorucen zu verwalten. Durch diese APIs können \ac{IaC}
Werkzeuge zur Automatisierung genutzt werden. Die Cloud Infrastruktur wird als
Code geschrieben, was es ermöglicht Software Engineering Best-Practices auch in
der Verwaltung von Infrastruktur zu nutzen. In der Literatur lässt sich vieles
unter dem Begriff "GitOps" dazu finden.

\subsubsection{Automatisierung durch Containerisierung}

%TODO

% Eventuell die Vorteile von Microservices an einem Beispeil zeigen?

% Eines der wichtigsten Vorraussetzungen ist die Größe einer Applikation. Wenn
% weniger Personen als ein Team (ca. 8-12 Personen) an einer Applikation arbeiten,
% ist der organisatorische und technische Overhead\footnote{TODO: Deutsches Wort}
% es nicht Wert.

% Ebenso entscheiden Faktoren wie Status der Applikation im Lebenszyklus, Stärke
% der IT-Standardisierung (Compliance, Richtlinien, und Governance), Außmaß der
% Automatisierung und welche Sprache\footnote{Herausfinden, warum ich Sprache
% geschrieben habe; Eventuell API und Datenbanken} für die existierende
% Applikation genutzt wurde.



% Kubenernetes/Container/Unikernel sind gut für IT-Automatisierung
