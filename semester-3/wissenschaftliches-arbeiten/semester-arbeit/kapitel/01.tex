\newpage
\section{Einleitung} \label{Einleitung}

Ein Trend in der Softwarearchitektur ist aktuell die
Microservice-Architektur\footnote{Citation Needed}. Dabei werden verschiedene
Komponenten, Services und Subsysteme einer Applikation identifiziert und seperat
entwickelt. Es gibt eine Vielzahl an wissenschaftlichen Arbeiten aus der
Industrie, die sich damit befassen, wie eine Migration von Monolithen zu
Microservices durchgeführt werden kann, doch gibt es wenig zu finden, um die
Frage zu beantworten, wann und zu welchen Umständen eine solche Migration Sinn
ergibt. Diese Arbeit fasst den aktuellen Forschungsstand und Erfahrungsberichte
aus der Industrie zusammen, um eine Antwort auf diese Frage zu finden.

\subsection{Forschungsfrage}
Unter welchen Umständen ergibt eine Migration von Monolithen zu Microservices Sinn?

- Einleitung
- Definition von Microservices und Unterschiede zu Monolithen
- Wann können Microservices genutzt werden?
- Wann ist es nicht möglich Microservices zu nutzen?
- Fazit

\subsection{Aufbau der Arbeit}
Im Kapitel \ref{grundlagen} werden Microservice Architekturen definiert und von
Monolithen differenziert. Diese Kapitel bietet die technische Grundlage für
diese Arbeit.

Kapitel \ref{wann-msa} befasst sich mit den Vorraussetzungen und Umständen zu
denen eine Nutzung von Microservices Sinn ergibt.

Im Kapitel \ref{wann-nicht-msa} wird aus den Vorherigen Kapiteln abgeleitet,
wann Microservices keinen Sinn geben.

Zuletzt werden die Ergebnisse dieser Semesterarbeit im Kapitel \ref{fazit}
zusammengefasst.
