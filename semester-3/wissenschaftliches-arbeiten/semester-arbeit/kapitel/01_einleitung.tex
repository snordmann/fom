\newpage
\section{Einleitung} \label{Einleitung}
IT-Systeme helfen dem Menschen, indem diese Prozesse automatisieren,
standartisieren und vereinfachen. Durch diese Macht der IT-Systeme muss man
diese aber auch vor dem menschlichem Versagen oder gar menschlichen Angriffen
schützen.
In dieser Arbeit werden verschiedene Arten des menschlichen Versagens
kategorisiert, inklusive Motivationen für Angreifer, und die Zielkonflikte
zwischen Benutzbarkeit und Sicherheit dargestellt.


\subsection{Forschungsfrage}
Wie können IT-System-Architekturen entworfen werden, sodass böswillige
Manipulation und menschliches Versagen von bestimmungsgemäßen Gebrauch unterschieden
werden können?

Differenzierung von böswilliger 

% Wie kann man IT-Systeme vor menschenlichen Versagen und böswilliger Manipulation
% schützen, sodass diese Systeme trotzdem vom den Menschen nutzbar bleiben?

\subsection{Aufbau der Arbeit}
Im Kapitel \ref{grundlagen} werden die Grundlagen der IT-Sicherheit, wie sie für diese
Arbeit relevant sind, kurz zusammengefasst. Ebenso wird hier auf den
Zielkonflikt zwischen Sicherheit und Nutzbarkeit durch den Menschen eingegangen.

Kapitel \ref{menschen} behandelt die Arten des menschlichen Versagens und Motivationen
für einen Angriff auf ein IT-System.

Im Kapitel \ref{massnahmen} werden Maßnahmen vorgestellt, wie ein Unternehmen sich vor den
verschiedenen Arten des Versagens schützen kann. Hierbei wird vor allem auf die
Umsetzbarkeit dieser Maßnahmen geachtet.

Zuletzt werden die Ergebnisse dieser Semesterarbeit im Kapitel \ref{fazit}
zusammengefasst.
